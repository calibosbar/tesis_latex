%%% File encoding is ISO-8859-1 (also known as Latin-1)
%%% You can use special characters just like �,� and �

\chapter{Introduction}

Why UGC11680NED01 spiral galaxy is red? 

With this simple question began work presented in this thesis.

In any textbook we learn that elliptical galaxies are red and spiral galaxies are blue, Two morphologies, two colors. 
We can distinguish between these types of galaxies without really studying very thoroughly, and go with something else.
However, not everything is so easy, people have discovered red spiral galaxies (Bundy) in the local universe. 
This simplification is justified because most galaxies follow a relationship strictest Color-morphology.
To study the nature of the red spiral, we will have many preconception off on their star formation. 
This has certain advantages in the analysis. Thus we will take the spiral of the catalog, and try to analyze their stellar populations.

The advent of large galaxy surveys like the Sloan Digital Sky
Survey (SDSS) in which photometry (and therefore color) is
readily available to millions of objects has led to
use of optical colors to define "early" galaxies "tardias``
(Eg Cooray 2005; Bundy et al 2006 ;. Croton Lee et al 2007 ;.
And pen 2007; Salimbeni et al. 2008; Simon et al. 2009). This method
is particularly favored because obtaining morphologies for a large
number of galaxies has been impossible until recently. This simplification
is justified because it has been shown many times that
Most galaxies follow a color ratio - strict morphology.
For example, Mignoli et al. (2009) showed that 85 percent of
galaxies to $ z \ approx $ 1 are either, red galaxies dominated by its bulb or
blue galaxies dominated by its disc; while Conselice (2006) showed
Similar result 22 000 galaxies of low redshift (both using automated
methods for morphological classification).
However, the clear correlation between the color and morphology
surprising given that the colors of the galaxies are determined primarily
for its stellar content (and therefore their recent star formation,
especially within the last Gyr), while morphology is mainly
driven by the dynamic history. The clear link between color
and morphology gives a strong indication that the time scales
and the processes that drive morphological transformation and
cessation of star formation are related (at least in most
cases). In this paper, however, we consider a class of object (red spiral)
where it seems that the relationship described above seems to be fulfilled.

Since the morphology-density relation quantified first
(Dressler, 1980), many mechanisms have been proposed for
Blue transformation, disk galaxies forming stars at low density
regions of the Universe, a, passive red galaxies in clusters first type.
A recent review of many of the proposed mechanisms and
the evidence supporting them, can be found at Boselli and Gavazzi
(2006). Clearly two things must happen for a blue star-forming spiral
Galaxy becomes a passive early red type. Education First Star
You must cease (which can alter the morphology indirectly cause
spiral arms and the disc generally disappear, possibly producing a
S0 lenticular or a spiral). Secondly, in order to produce a bona
fide elliptical, the same or a different process must also dynamically
alter the stellar kinematics of the galaxy.

The presence of an unusual red colored or passive (ie, not forming star)
population of spiral galaxies in clusters of galaxies was first
said Van den Bergh (1976) in the Virgo cluster. later studies
distant cluster of galaxies in the Hubble Space Telescope (HST)
images also revealed a number of so-called "passive"
spiral galaxies with a lack of ongoing star formation (Couch et al.
1998; Dressler et al. 1,999; Poggianti et al. 1,999). Passive delayed type
galaxies were identified on the move to lower red outside SDSS
clusters of Goto et al. (2003) using the concentration as a proxy for
morphology. Spirals liabilities of a cluster at z ~ 0.4 were studied by
Moran et al. (2006) who found stories of star formation in the Galaxy Evolution Explorer (GALEX) consistent with observations
closing of star formation choke (as described
by Bekki, sofa and Shioya 2002). Spirals liabilities have also revealed
in a cluster at z ~ 0.1 in the A9012 / Galaxy Space Telescope
Evolution Survey (STAGES) using HST morphologies Wolf et al.
Spectral energy distributions (2009), rest-frame near-ultraviolet optical
(SED; Wolf, Gray \& Meisenheimer 2005) and data from 24 microns
Spitzer (Gallazzi et al. 2009). In this series of papers, 'dusty
are red late types "and" optically passive final rates "to be
largely the same, with a non-zero (but reduced significantly)
rate of star formation revealed by infrared data.
Spiral red / final year types have been studied in several recent articles
(Lee et al 2008 ;. Cortese and Hughes 2009; Deng et al 2009 ;. Hughes
And Cortese 2009) and Mahajan and Raychaudhury (2009) who
speak passive blue galaxies (ie, galaxies with blue, but
showing signs of recent star formation in their spectra)
most of which seem to have late-type morphologies and have very
Recently off star formation. These could be the progenitors
red spirals. Bundy et al. (2010) studied the redshift
evolution of disk galaxies red string as components in the
Cosmic Evolution Survey (COSMOS) and use it to estimate that
up to 60 percent of spiral galaxies must go through this
phase on the way to the red string - making it an important
evolutionary step.

It is clear that all spiral galaxies can be affected by various physical
processes as they evolve - in this paper seeks to identify
that they are most important for red spirals, asking how are
able to shut down star formation, keeping their spiral morphology.
A list of possible mechanisms include processes
It will depend on the environment. (1) galaxy-galaxy interactions: high density
regions, there is a greater probability of interaction
with other galaxies. Most major mergers destroy spiral structure
(Toomre and Toomre 1972) unless they involve parents in rich gas
(Hopkins et al. 2009), but some interactions can be quite
soft (for example, Walker and Hernquist Mihos 1996), for example minormergers,
tidal interactions, etc. (2) The interaction with the cluster itself
also it produced and can be removed form the gas reservoir
star formation. This may be due to tidal effects (eg Gnedin 2003)
or interaction with the intercluster hot gas, either through thermal evaporation (Cowie and Songaila 1977) or ram pressure extraction
(Gunn and Gott 1972). (3) The processes as harassment (Moore et al.
1999) and the starvation and strangulation (Larson, Tinsley and Caldwell
1,980; Bekki et al. 2002) have also shown to have a significant
cant effect in late-type galaxies. Harassment refers to heating
gas for many small interactions, while hunger or strangulation
It refers to the gradual depletion of disk after the hot gas has halo
It has been stripped away. Both of these mechanisms occur in much greater
radios cluster (ie, lower densities) than the "classic environmental effects.
Internal mechanisms may be more important. For instance,
(4) the last semi-analytic models of galaxy formation invoke all
feedback from a massive black hole active center [or galactic nuclei
(AGN)] to explain the most massive elliptical galaxies red (Granato
et al. 2004; Silk 2005; Schawinski et al. 2006; Croton et al. 2006;
Bower et al. 2,006), although the effect of this process on disk galaxies
has been studied less, you can still have some effect (Okamoto,
Nemmen and Bower 2008). (5) Another cause could be the instability bar
in spiral galaxies that lead the gas into (eg Combes
Sanders 1981), and may trigger the AGN activity and / or central star
training (eg Shlosman, Peletier and Knapen 2000), perhaps using
the gas reservoir in the outer disk and spirals making red. (6)
Finally, spirals Red spirals could simply be old who have exhausted
all its gas in normal star-forming activities without any
basic interaction. In normal spirals, gas supplying current
star formation comes from material falling from a reservoir in the
outer halo (Boselli and Gavazzi 2006). As first it suggested by Larson,
Tinsley and Caldwell (1980) and expanded by Bekki et al. (2002),
extracting gas from this outer halo ('throttle' or 'hungry')
will cause the gradual cessation of star formation proceeding more
several Gyr.




The quenching of star-formation in massive galaxies is one of the most fundamental aspects of galaxy evolution in the last ~8 Gyrs that is still poorly understood. It should involve both a ?switching-off? of the star formation and a morphological transformation: (1) To reconcile the predicted and observed massive end of the galaxy luminosity function; (2) To understand why during this cosmological time (from z~1 to the present time), most of the stars are formed in late-type (blue-cloud) galaxies (e.g., Wolf et al. 2005), but the mass is accumulated in early-type (red-sequence) ones (e.g., McIntosh et al. 2005). (3) To reproduce the observed ratio of red-sequence to blue-cloud galaxies at different cosmological times.
Standard explanations invoke galaxy mergers which can drive inflows of cold gas that fuel central starbursts and AGNs (e.g., Hopkins et al., 2008). 
The gas is then consumed by the starburst, and subsequent outflows from the AGN (and the starburst itself), can heat (and spell-out) the remaining gas, quenching the starformation and switching off the nuclear activity. The cosmological co-evolution of the black-hole/bulge masses (eg., Cisternas et al. 2011), and the location of the AGN hosts in the so-called transition Green-Valley region of the CM-diagram (e.g., Kauffman et al. 2003; Sanchez et al. 2004), support this scenario.
However, observations have struggled to test the details of this ?self-regulated? hypothesis. On one hand there is a lack of direct evidence of the relation between the merging processes and the AGN activity in general. This may be related to the differences in the time-scale associated with both processes. On the other hand, there are some puzzle examples of apparently ?quenched? galaxies that does not fit with this scenario. One example is the significant population of bar-dominated (face-on) disk galaxies on (or above) the red-sequence observed at different redshifts (e.g, Masters et al. 2010; Bundy et al. 2010): How can a merger shut off star-formation without destroying the disk? Host these galaxies and active nuclei? Could they fit within the standard explanations at all?
Three complementary scenarios have been proposed: 
(a) Disks are re-grown in gas-rich mergers before quenching completes, perhaps during the starburst phase (e.g., Robertson et al. 2006); 
(b) Quenching may also be driven by ?starvation? in average high-density environments (e.g., Bekki, 2009); 
(c) Star-formation is suppressed by internal structural changes (e.g., ?morphological quenching?, Martig et al. 2009). 

Integrated measures cannot distinguish between a truncated disk model in which SFH has gradually ended over 2 Gyrs (scenario b and c), 
and a 2-component model in which a central starburst (20\% by mass) lead to a rapid quenching (scenario a). 
However, spatially resolved star-formation histories and gas metallicity enrichment gradients, 
like the ones provided by CALIFA, are a robust measure of the burst fraction in these proposed transitioning galaxies.

Due to the scarce number of face-on red spiral galaxies most probably CALIFA will not sample a representative number of them. 
However, in the first observing run we observe one: UGC11680. This galaxies is a M51-like galaxy, with a nearby companion under interaction. 
It has a Seyfer 2 nuclei spectra, and it has been clasified as one of the reddest Seyfert galaxies in different NIR surveys. 
It u-z color place is above the red-sequence.

Although it is just one object, we can test on it the different scenarios proposed for face-on spiral red objects, 
by performing a comparison of its spatilly resolved spectroscopic properties with those of face-on spiral galaxies of a similar luminosite (mass).
This is an unique property of CALIFA, the capability of providing suitable comparison samples for ?rare? objects.




The insight that our Milky Way is just one of many galaxies
in the Universe is less than 100 years old, despite the fact
that many had already been known for a long time. The
catalog by Charles Messier (1730-1817), for instance, lists
103 diffuse objects. Among them M31, the Andromeda
galaxy, is listed as the 31st entry in the Messier catalog. Later,
this catalogue was extended to 110 objects. John Dreyer
(1852-1926) published the New General Catalog (NGC) that
contains nearly 8000 objects, most of them galaxies. Spiral
structure in some of the nebulae was discovered in 1845 by
William Parsons, and in 1912, Vesto Slipher found that the
spiral nebulae are rotating, using spectroscopic analysis. But
the nature of these extended sources, then called nebul\ae, was
still unknown at that time; it was unclear whether they are
part of our Milky Way or outside it.


\section{The nature of the nebul\ae}. The year 1920 saw a public
debate (the Great Debate) between Harlow Shapley and
Heber Curtis. Shapley believed that the nebul\ae are part
of our Milky Way, whereas Curtis was convinced that the
nebul\ae must be objects located outside the Galaxy. The
arguments which the two opponents brought forward were
partly based on assumptions which later turned out to be
invalid, as well as on incorrect data. Much of the controversy
can be traced back to the fact that at that time it was not
known that dust in the Galactic disk leads to an extinction
of distant objects. We will not go into the details of their
arguments which were partially linked to the assumed size
of the Milky Way since, only a few years later, the question
of the nature of the nebul\ae was resolved.

In 1925, Edwin Hubble discovered Cepheids in
Andromeda (M31). Using the period-luminosity relation for
these pulsating stars he derived a distance
of 285 kpc. This value is a factor of 3 smaller than
the distance of M31 known today, but it provided clear
evidence that M31, and thus also other spiral nebul\ae, must
be extragalactic. This then immediately implied that they
consist of innumerable stars, like our Milky Way. Hubble's
results were considered conclusive by his contemporaries and marked the beginning of extragalactic astronomy. It is
not coincidental that at this time George Hale began to
arrange the funding for an ambitious project. In 1928 he
obtained six million dollars for the construction of the 5 m
telescope on Mt. Palomar which was completed in 1949.


 

The road to understanding the processes that lead to the evolution of galaxies has
It has been arduous. It is surprising that until the 1920s did not know of the existence of other galaxies than
the Milky Way. The hitherto prevailing view was that the universe consisted of the Milky Way
and a vacuum around it. Four hundred years ago, Galileo Galilei turned his telescope on the Milky
Way and discovered that consisted of countless faint stars that are not visible to the naked eye.
Nearly 150 years later, the philosopher Immanuel Kant speculated that gravity should act between
stars of the Milky Way in the same way that gravity is responsible for movements
Solar System planets, and other observed nebulae could be similar to ours but
far away (i.e., would be  island Universes). 

More than 90 years since the so-called Great Debate between Harlow Shapley and Heber Curtis 
settled the true nature gaseous nebulae. The debate was finally ended by Edwin Hubble in 1923, 
measuring the distance to Andromeda using the Cepheid distance and that has proved that these spiral nebulae are in
Indeed, entire galaxies outside our Milky Way (Hubble 1925).Since then,
the concept of these objects, galaxies or universes island located beyond the Milky Way,
began to spread.

Moreover, it is worth mentioning that until the 1980s were considered the stars
as the dominant form of matter in the Universe. Thus, new theoretical ideas put to
It is shown that dark matter originally discovered by Zwicky (1933) could be
neutral elementary particles not baryonic (Cowsik \& McClelland 1973; Peebles, 1982) while
X-ray images showed that most baryons in rich clusters is in the
form of hot intergalactic gas (Forman \& Jones 1982). The result of our best knowledge
the properties of galaxies became clear that the baryons are not the dominant form of matter
in our universe, and that the stars represent only a small fraction of the baryons (eg
Fukugita, Hogan \& Peebles 1998) .Despite the significant progress in recent years
in our understanding of the baryon physics, formation and evolution of galactic discs
still remain as two of the most important aspects to understand in full, within the field of astronomy
extragalactic.

The current image on the formation of galaxies in general is based on the model
Hierarchical clustering of Cold Dark Matter (within the standard paradigm of $ \Lambda $ CDM), the
which attempts to explain how the structures we see today were formed as a result
the growth of primordial fluctuations (gravitational instabilities) after a period
when the Universe was extremely homogeneous, as it exemplifies the Cosmic Background
Microwave. According to the model $ \Lambda $ CDM most of the matter in the universe is
in the form of (non-relativistic) cold baryonic matter not subject to gravitational interactions and whose
virialization cooling process takes place without the emission of photons. The latest results
of the Planck project confirms that our universe has an age of $ 13,798 \pm $ 0.037 million years
and it consists for $ 4.82 \pm $ 0.05 \% ordinary baryonic matter, 25 $ \pm $ 4 \% dark matter and
$ 69 \pm $ 1 \% dark energy (Planck Collaboration et al. 2015). Of particular interest is the fact that,
paradigm $ \Lambda $ CDM galaxies represent only the tip of the iceberg of a Universe dominated
for some unknown dark matter, and an even harder way to study: dark energy.
Thus, understanding the wealth of morphologies, sizes and luminosities of the galaxies within
a cosmological context is a task of great importance but of no less difficulty (Mo, Mao \&
White 1998 onwards MMW).


Most of the visible matter in the Universe is concentrated in galaxies, which are the basic ecosystems in astronomy in which stars form, evolve and die for
a process which keeps in constant interaction with the interstellar medium (ISM). Galaxies
also they represent beacons that allow us to explore our Universe to cosmological scales.
From the pioneering work of Edwin Hubble (1926b, 1936), which was who first proposed
morphological classification system (called the Hubble diagram) for galaxies, knows-
we that the universe is populated by different types of galaxies are arranged in three categories
General according to the form presented (originally on photographic plates): elliptical, spiral and
irregular; the first type galaxies are relatively rounded in shape and are made
by a large number of stars with a distribution of triaxial movement, while the pins
ral flat discs are dominated almost entirely by the prescribed rotations. 

Today, this has a more complex classification taxonomy and continuously updated using not only the
information of the optical bands, but also other wavelengths. Other types of galaxies
galaxies are low surface brightness dwarf galaxies, ultra-faint galaxies, spheroidal and
galaxies in transition between subclasses (often due to environmental effects). Figure 1.1
illustrates how the Hubble diagram for classifying galaxies would be at different times
history of the Universe.


The standard model $\Lambda $ CDM (Springel et al. 2006) provides a framework that can
scale to understand the main mechanisms of formation and evolution of structure
the universe. However, the formation of disk galaxies has proved particularly
hard to understand. In the context $ \Lambda $ CDM galaxies are systems consisting of a structure
star embedded in a halo of dark matter mass growing from hierarchical clustering
lower mass halos. However, understanding the evolution of the baryonic component
under this scenario for hierarchical dark matter it is not yet complete. Thus, it is well known that
photometric, chemical, and kinematics of galaxies today are the result of properties
more complex mechanisms such as the initial conditions that were formed and
the interaction between internal and external processes, both fast and secular. Although difficult
separate the effects of all these mechanisms, any theory that seeks to explain satisfactorily
factory formation and evolution of galaxies disk must be able to account for all
these processes.

The most common types of galaxies in a limited exploration in magnitude within the Universe
Local would spiral galaxies. These represent about 77 \% of all galaxies obtained
servadas (20 \% are elliptical and 3 \% irregular, Li \& White 2009). In this doctoral thesis
we will pay special attention to the spiral discs observed within exploration CALIFA
(Calar Alto Legacy Integral Field Area Survey) 1 which also represents the most common type of ob-
I ject observed as part of this project (see Chapter �4 and �5). Disk galaxies consist
a disk component made up of stars, dust and cold gas (both atomic and molecular)
a central component of bulb, a stellar halo, and a halo of dark matter. Often they appreciated
spiral arms and also a high percentage of spirals also have a component
bar-shaped core. Their typical stellar masses ranging from $10^9$ and $10^12$ $M _{\odot}$,
their luminosities between $10^8$ and $ 10^11  L_{\odot}$, sizes between 5 and 100 kpc in diameter,
rotation speeds of about 200-300 km / s scales disc of about 4 ksi.
His records often can be separated into a
Thin and thick component. The small disc is composed of young stars, while
the thick disc contains significantly less mass and their stars are older, richer in
metals and are dynamically hotter. Since our own galaxy is a spiral (barred)
most of our understanding of the formation of spiral galaxies comes from studies
made on the various components of the Milky Way.


As mentioned previously, many aspects of the formation and evolution
galaxy and especially on the evolution of spiral galaxies remain unexplained.
They were White \& Rees (1978) who first raised in the formation of these galaxies
as a two-step process whereby disk galaxies form by contraction dissipate
tory within halos of dark matter. On the one hand, the structures of dark matter would
hierarchically grown collapsing in individual halos with a certain amount of time prior
acquired regulate tidal pairs. Furthermore, the gas initially follow the same evolution
dark matter until dissipation emission light of the first stars let their
virialization contraction and increasingly smaller sizes provided that could transfer
total angular momentum of the object (which is not altered by light emission) to a component
relatively large disk. This process virialization increasingly smaller radii and training
a thin disk help to increase both the gas density, facilitating a process of
fragmentation and star formation in the resulting molecular clouds. Historically, pro-
They put two scenarios for the formation of the component disc (see White \& Rees 1978; Fall
\& Efstathiou 1980). The first contemplated a monolithic collapse of a large gas cloud Tama
No, where the disc is formed by conservation of angular momentum (Eggen et al. 1962), while
the second mechanism is based on the coalescence of smaller progenitors with some
total angular momentum also end up stored in a stellar disk (Searle \& Zinn 1978).


Today the most widely accepted scenario for the formation of discs (White \& Frenk 1991; MMW)
It is known as the inside-out stage collapsing combining some initial spherical with a
stage hierarchical mergers of halos which would further lead to the formation of a thick disk
and an effective redistribution of angular momentum component in the thin disk. This scenario
predicts a direct sequence of events beginning with the initial formation of a bulge
due to high densities and times of rapid cooling of the innermost part of a galaxy.
Basically, a halo of dark matter that provides a potential gas cloud protoga-
rotating lactic increase its density and collapses by gravity. During the collapse,
gas is cooled through radiative processes until equilibrium is reached and virializa. By
Furthermore, the dark matter halo gains angular momentum through tidal pairs generated
by the large-scale structure around it and the numerous mergers that take place during
this first stage. Due to conservation of angular momentum that results in the formation of a
relatively thick disk. But until that it does not diminish the pace of mergers (a $ z <1 $) is not
I began to form a thin disk where the bulk of star formation occurs (due
Jeans instabilities in which the balance between self-gravity and thermal pressure breaks)
and further wherein the transfer of angular momentum is more effective because of its abundance of gas.

The conservation of angular momentum in this case leads to the formation of surface profiles exponential mass density (Freeman 1970).
A quantitative analysis of this model and
different theories on the formation of discs described below, although the details of their
modeling going beyond the scope of this thesis and the reader is referred to the original papers and / or
the classic book of Binney \& Tremaine (1987 or subsequent updates) for review
Detailed.

Qualitatively, according to inside-out stage, a disc is formed with properties
similar to those observed only if the gas retains most of its angular momentum; in a first
mere stage halos non-baryonic dark matter are formed from primordial fluctuations,
then it condenses the gas cools in these halos. Its main assumptions are: (I)
mass and angular momentum of the disk represent a fixed fraction of the mass and angular momentum
halo of dark matter; (II) it is a thin disc structure supported by rotation and with a profile
exponentially surface gloss; (III) only dynamically stable systems may correspond to
discs real galaxies; (IV) the internal structure of the halo is assumed to follow a profile of dental
sity Navarro, Frenck \& White (1997); (V) halo assuming ball is maintained during
the collapse. In this context, it would be forming stars from gas from the beginning of the
disk formation, from gravitational instabilities that lead to cloud formation
molecular giants (Elmegreen \& Elmegreen 1983). Disc stability can be ensured and
either through the pressure when the local dynamic time scale is larger than the scale of
associated with the velocity of sound in the gas, or by the velocity dispersion time
the stars when the period of epicycles is less than the time scale of local dynamics. This
It is known as the stability criterion Toomre (1964), which establishes a relationship between
parameters of a gaseous disk with differential rotation and which is applicable in linear approximation, it is
ie, away from the resonances between the spiral pattern and the rotational motion (co-rotation) and epi-
cyclical (ILR and OLR) of stars. This criterion applies only to the stability against perturbations
axisymmetric and is also used to explain the presence of thresholds star formation
the outer regions of the disks of galaxies.


